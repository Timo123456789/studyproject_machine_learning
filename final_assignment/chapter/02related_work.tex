%!TEX root = ../Final_Assignment_SP_ML4IM_2023.tex
\chapter{Related Work}
\label{ch:related_work}

Insect monitoring is a field that invokes a lot of interst in the scientific community. Therefore a lot of studies were conducted in the past. 

A study on automated monitoring was able to achive good results using only one camera that was connected to a computer and a deep learning algorithm. In this setup the images were captured and delivered to the computer for preprocessing. After that the deep learning algorithm was used to analyze these images. The results were promising, achiving a confidence level of at least 70\% \citep{mendoza2023}. 

Therefore this model seems to be a effective low-cost solution for monitoring insects. The study also showed concerns for the model to be used in the field, as the manuel focus of the camera would lead to massiv data losses \citep{mendoza2023}. 

Another study detecting small traffic signs achived great results using the YOLOv7 algorithm. It was shown that the model was able to achive a MaP@0.5 of 88.7\% for the improved YOLOv7 algorithms. But also the basic algorithm achive a MaP@0.5 of 83.2\% \citep{li2023}. 

This shows that the there were good results with simple models for insect detection and also that the YOLO algorithm is a good choice for object detection, therefore it will also be used in this study. 