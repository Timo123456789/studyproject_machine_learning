%!TEX root = ../Final_Assignment_SP_ML4IM_2023.tex
\chapter{Introduction}
\label{ch:introduction}

% Motivation etc.
% kurz

This final assignment of the course "Study Project: Machine Learning meets Insect Monitoring" at the Institute for Geoinformatics at the University of Münster held by Prof.\ Dr.\ Benjamin Risse and Sebastian Thiele is about the integration and development of a machine learning model for the detection of insects in videos captured by a in-house constructed camera trap. The camera trap is part of the Computer Vision and Machine Learning Systems Group of Prof.\ Dr.\ Benjamin Risse and it delivered the data for this project \cite{ComputerVisionAndMachineLearningSystems}.

Insects are an important part of the ecosystem and are essential for the pollination of plants, the food chain and pest control. However, the populations of insects are drastically declining \cite{wagner2021insect}. To understand the causes of the decline and to develop strategies to protect insect populations, monitoring is important. Monitoring is challenging because insects are small and difficult to observe and identify. Valide data and time series are needed to understand the causes of the decline and to understand the actual state of the insect populations. Current monitoring methods are time-consuming and expensive, e.g. malaise traps or light traps. Therefore, the development of a camera trap for insects is a promising approach to monitor insect populations.

The study project goal is to train models, which can detect insects in videos, with different strategies, to evaluate the performance of the models and find the best methods for insect detection. For this endeavour, this group used the RGB stream of the camera trap and developed different approaches to preprocess the videos to get a better detection of the tiny insects.

After outlining the motivation and goal of this project in the first chapter, the current state of research in the field of insect detection and monitoring is presented in chapter \ref{ch:related_work}. The fundamentals of the machine learning models used and the computer vision library are introduced in chapter \ref{ch:fundamentals}. In chapter \ref{ch:methods} the methods used in this project are explained. The results are discussed in the chapter \ref{ch:discussion}. The contributions of the group members are summarized in the chapter \ref{ch:contributions}.
